%%  ************    LibreSilicon's StdCellLibrary   *******************
%%
%%  Organisation:   Chipforge
%%                  Germany / European Union
%%
%%  Profile:        Chipforge focus on fine System-on-Chip Cores in
%%                  Verilog HDL Code which are easy understandable and
%%                  adjustable. For further information see
%%                          www.chipforge.org
%%                  there are projects from small cores up to PCBs, too.
%%
%%  File:           StdCellLib/Documents/LaTeX/LATESP_manpage.tex
%%
%%  Purpose:        Manual Page File for LATESP
%%
%%  ************    LaTeX with circdia.sty package      ***************
%%
%%  ///////////////////////////////////////////////////////////////////
%%
%%  Copyright (c) 2019 by chipforge <stdcelllib@nospam.chipforge.org>
%%  All rights reserved.
%%
%%      This Standard Cell Library is licensed under the Libre Silicon
%%      public license; you can redistribute it and/or modify it under
%%      the terms of the Libre Silicon public license as published by
%%      the Libre Silicon alliance, either version 1 of the License, or
%%      (at your option) any later version.
%%
%%      This design is distributed in the hope that it will be useful,
%%      but WITHOUT ANY WARRANTY; without even the implied warranty of
%%      MERCHANTABILITY or FITNESS FOR A PARTICULAR PURPOSE.
%%      See the Libre Silicon Public License for more details.
%%
%%  ///////////////////////////////////////////////////////////////////
\label{LATESP}
\paragraph{Cell}
\begin{quote}
    \textbf{LATESP} - a High-active D-Latch with high-active Clock Enable and low-active asynchronous Set
\end{quote}

\paragraph{Synopsys}
\begin{quote}
    LATESP(Q, D, SN, E, X)
\end{quote}

\paragraph{Description}
%%  ************    LibreSilicon's StdCellLibrary   *******************
%%
%%  Organisation:   Chipforge
%%                  Germany / European Union
%%
%%  Profile:        Chipforge focus on fine System-on-Chip Cores in
%%                  Verilog HDL Code which are easy understandable and
%%                  adjustable. For further information see
%%                          www.chipforge.org
%%                  there are projects from small cores up to PCBs, too.
%%
%%  File:           StdCellLib/Documents/LaTeX/LATESP_circuit.tex
%%
%%  Purpose:        Circuit File for LATESP
%%
%%  ************    LaTeX with circdia.sty package      ***************
%%
%%  ///////////////////////////////////////////////////////////////////
%%
%%  Copyright (c) 2019 by chipforge <stdcelllib@nospam.chipforge.org>
%%  All rights reserved.
%%
%%      This Standard Cell Library is licensed under the Libre Silicon
%%      public license; you can redistribute it and/or modify it under
%%      the terms of the Libre Silicon public license as published by
%%      the Libre Silicon alliance, either version 1 of the License, or
%%      (at your option) any later version.
%%
%%      This design is distributed in the hope that it will be useful,
%%      but WITHOUT ANY WARRANTY; without even the implied warranty of
%%      MERCHANTABILITY or FITNESS FOR A PARTICULAR PURPOSE.
%%      See the Libre Silicon Public License for more details.
%%
%%  ///////////////////////////////////////////////////////////////////
\begin{center}
    Circuit
    \begin{figure}[h]
        \begin{center}
            \begin{circuitdiagram}{25}{17}
            \pin{1}{1}{L}{X}    % pin X
            \pin{1}{5}{L}{E}    % pin E
            \usgate
            \gate{nand}{5}{3}{R}{}{}
            \wire{9}{3}{11}{3}
            \junct{10}{3}
            \wire{10}{3}{10}{7}
            \gate{not}{14}{3}{R}{}{}
            \wire{17}{3}{18}{3}
            \wire{18}{3}{18}{7}
            \pin{1}{9}{L}{D}    % pin D
            \wire{2}{9}{7}{9}
            \gate[\tristate{Dp}]{not}{10}{9}{R}{}{}
            \wire{13}{9}{15}{9}
            \junct{14}{9}
            \wire{14}{9}{14}{12}
            \gate[\tristate{Dp}]{not}{18}{9}{L}{}{}
            \pin{13}{16}{L}{SN}  % pin SN
            \gate{nand}{17}{14}{R}{}{}
            \wire{21}{9}{23}{9}
            \junct{22}{9}
            \wire{22}{9}{22}{14}
            \wire{21}{14}{22}{14}
            \pin{24}{9}{R}{Q}   % pin Q
            \end{circuitdiagram}
        \end{center}
    \end{figure}
\end{center}

%\input{LATESP_schematic.tex}

\paragraph{Truth Table}
%\input{LATESP_truthtable.tex}

\paragraph{Usage}

\paragraph{Fan-in / Fan-out}

\paragraph{Layout}

\paragraph{Files}

\clearpage
